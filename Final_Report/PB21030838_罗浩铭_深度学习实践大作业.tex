
\documentclass[twoside,12pt]{article}
%\documentclass[UTF8]{ctexart}
\usepackage[heading=true]{ctex}

\RequirePackage{natbib}
% modification to natbib citations
\setcitestyle{authoryear,round,citesep={;},aysep={,},yysep={;}}

\usepackage{listings}
\usepackage{color}

\definecolor{dkgreen}{rgb}{0,0.6,0}
\definecolor{gray}{rgb}{0.5,0.5,0.5}
\definecolor{mauve}{rgb}{0.58,0,0.82}

\lstset{frame=tb,
  language=Python,
  aboveskip=3mm,
  belowskip=3mm,
  showstringspaces=false,
  columns=flexible,
  basicstyle={\small\ttfamily},
  numbers=none,
  numberstyle=\tiny\color{gray},
  keywordstyle=\color{blue},
  commentstyle=\color{dkgreen},
  stringstyle=\color{mauve},
  breaklines=true,
  breakatwhitespace=true,
  tabsize=3
}

\usepackage{fancyhdr} % 页眉页脚
\usepackage{graphicx}
\usepackage{amsmath}
\usepackage[colorlinks=true, allcolors=blue]{hyperref}
\usepackage{geometry}

\geometry{
  paper      = a4paper,
  vmargin    = 2.54cm,
  hmargin    = 3.17cm,
  headheight = 0.75cm,
  headsep    = 0.29cm,
  footskip   = 0.79cm,
}

\newcommand{\update}[1]{{\textcolor{black}{#1}}}
\newcommand{\boldres}[1]{{\textbf{\textcolor{red}{#1}}}}
\newcommand{\secondres}[1]{{\underline{\textcolor{blue}{#1}}}}

\pagestyle{fancy}

%\firstpageno{1}

\title{ }

\author{罗浩铭\ PB21030838}


\begin{document}

\fancyhf{} % 清除所有页眉页脚
\fancyfoot[C]{\thepage} % 设置右页脚为页码
\fancyhead[l]{\footnotesize  }
% 设置右页眉为章节标题 

\renewcommand{\headrulewidth}{0pt} % 去页眉线

\begin{center}
  \textbf{\LARGE{深度学习实践大作业——iTransformer}}\\
  \vspace{0.2cm}
  \large{罗浩铭\ PB21030838}
\end{center}
% 无错别字,语句通顺,格式整齐,排版良好  (5分)

\section{文章基本信息}

\begin{itemize}
  \item 论文题目:iTransformer: Inverted Transformers Are Effective for Time Series Forecasting~\citep{itransformer}
  \item 论文作者:Yong Liu, Tengge Hu, Haoran Zhang, Haixu Wu, Shiyu Wang, Lintao Ma, Mingsheng Long
  \item 论文来源:ICLR 2024(在投,但在OpenReview已经拿到3个8分和一个6分,基本确定中会,参见https://openreview.net/forum?id=JePfAI8fah)
  \item 首发日期:2023年10月10日
  \item 论文链接:\url{https://arxiv.org/abs/2310.06625}
  \item 论文官方实现:\url{https://github.com/thuml/iTransformer}(截至本报告完成时,已获得443颗星)
\end{itemize}



\section{文章介绍及理解}
% 简述对文章的理解   300-600字     (10分)

我们选择的文章是,我们选择的文章是,我们选择的文章是,我们选择的文章是,我们选择的文章是,我们选择的文章是,我们选择的文章是,我们选择的文章是,

\section{代码结构}
% 对代码结构进行描述 300-1000字    (10分)

代码以\verb |run.py|为入口,其拉起experiments部分中的主类,主类再调用data_provider、model(model再调用layers)、utils等部分的类,完成数据的读取、模型的构建、训练、测试等过程。


\section{训练及测试过程}
% 对训练和测试过程进行描述(各种超参),需含loss曲线的展示,对时间的描述等   500-1000字	(15分)

我们同时使用百度的aistudio(每日4小时额度,单卡V100)和kaggle平台(每周30小时额度,单卡P100)进行训练。



\section{复现结果}
% 复现结果与文章所示结果的对比,并分析结果不同的可能的原因  500-1000字    (15分)
\subsection{模型效果测试}



\subsection{不同超参对模型效果的影响}



\subsection{对优化效率的iFlashTransformer的测试}





\section{感想}
% 复现过程中的感悟、吐槽  (300-1000字) (5分)




\section{探索性修改}
% (非必须,奖励分) 对代码进行了探索性修改,描述修改经过,并报告实验结果   (5分)


\bibliography{dllab_final_report}
\bibliographystyle{iclr2024_conference}

\vfill

\end{document}
