
\documentclass[twoside,12pt]{article}
%\documentclass[UTF8]{ctexart}
\usepackage[heading=true]{ctex}

\usepackage{listings}
\usepackage{color}

\definecolor{dkgreen}{rgb}{0,0.6,0}
\definecolor{gray}{rgb}{0.5,0.5,0.5}
\definecolor{mauve}{rgb}{0.58,0,0.82}

\lstset{frame=tb,
  language=Python,
  aboveskip=3mm,
  belowskip=3mm,
  showstringspaces=false,
  columns=flexible,
  basicstyle={\small\ttfamily},
  numbers=none,
  numberstyle=\tiny\color{gray},
  keywordstyle=\color{blue},
  commentstyle=\color{dkgreen},
  stringstyle=\color{mauve},
  breaklines=true,
  breakatwhitespace=true,
  tabsize=3
}

\usepackage{fancyhdr} % 页眉页脚
\usepackage{graphicx}
\usepackage{amsmath}
\usepackage[colorlinks=true, allcolors=blue]{hyperref}
\usepackage{geometry}

\geometry{
  paper      = a4paper,
  vmargin    = 2.54cm,
  hmargin    = 3.17cm,
  headheight = 0.75cm,
  headsep    = 0.29cm,
  footskip   = 0.79cm,
}



\pagestyle{fancy}

%\firstpageno{1}

\title{ }

\author{罗浩铭\ PB21030838}


\begin{document}

\fancyhf{} % 清除所有页眉页脚
\fancyfoot[C]{\thepage} % 设置右页脚为页码
\fancyhead[l]{\footnotesize  }
% 设置右页眉为章节标题 

\renewcommand{\headrulewidth}{0pt} % 去页眉线

\begin{center}
    \textbf{\LARGE{深度学习实践大作业——iTransformer}}\\
    \vspace{0.2cm}
    \large{罗浩铭\ PB21030838}
\end{center}

\section{文章选择及介绍}
我们选择的文章是,我们选择的文章是,我们选择的文章是,我们选择的文章是,我们选择的文章是,我们选择的文章是,我们选择的文章是,我们选择的文章是,

\section{代码结构}


\section{训练及测试过程}


\section{复现结果}


\section{感想}


\section{探索性修改}

\end{document}
